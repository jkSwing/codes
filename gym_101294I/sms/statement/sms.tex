\begin{problem}{СМС}{sms.in}{sms.out}{2 секунды}

%Автор задачи: Павел Кротков
%Автор условия: Павел Кротков

Одним из самых популярных способов использования мобильной связи являются СМС-сообщения. Каждый день миллионы людей отправляют десятки миллионов сообщений: <<Привет!>>, <<Как дела?>>, <<Ты где?>>, <<Я 
задержусь>> и еще тысячи различных коротких посланий.

К сожалению, на клавиатуре мобильного телефона может оказаться меньше кнопок, чем букв в алфавите. Поэтому на первой кнопке размещаются несколько первых букв алфавита, на второй~--- следующие несколько и так 
далее. Чтобы набрать некую букву, необходимо нажать ту кнопку, на которой она размещена, $t$ раз, если буква является $t$-ой по алфавиту, размещенной на этой кнопке. Так, если на первой кнопке клавиатуры 
размещены буквы \texttt{a}, \texttt{b} и \texttt{c}, то для выбора буквы \texttt{c} придется нажать эту кнопку трижды.

Ученые изучили среднее количество раз, которое каждая буква алфавита встречается в СМС-ках за время службы среднестатистического телефона. Теперь у фирмы-производителя появилась возможность спроектировать 
клавиатуру так, чтобы минимизировать суммарное количество нажатий на все кнопки. Помогите им сделать это.

\InputFile
Первая строка входного файла содержит два целых числа $n$ и $m$ ($1 \le n \le 30$, $1 \le m \le 10$)~--- количество букв в алфавите и кнопок на клавиатуре телефона. 

Следующая строка содержит $n$ целых чисел $a_i$ ($0 \le a_i \le 1{\,}000{\,}000$)~--- количество раз, которые будет напечатана $i$-ая буква алфавита.

\OutputFile
В выходной файл выведите $m$ целых неотрицательных чисел $b_i$~--- количество букв, отнесенных к $i$-ой кнопке клавиатуры. Сумма всех $b_i$ должна быть равна $n$. 

Если возможных ответов несколько~--- выведите любой.

\Examples
\begin{example}%
\exmp{
5 3
1 3 5 1 1
}{
1 1 3
}%
\end{example}

\Note
В примере на первой кнопке размещена одна первая буква алфавита, на второй~--- только вторая, а на третьей~--- оставшиеся три.

Таким образом, суммарное количество нажатий на кнопки будет:
$1 \times 1 + 3 \times 1 + 5 \times 1 + 1 \times 2 + 1 \times 3 = 14$.

\end{problem}
